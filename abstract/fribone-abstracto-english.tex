\chapter*{Abstract}

Companies like LG have introduced smart fridges in the last years; fridges that can be controlled via smartphone and allow know the information about the products what contains inside thanks to smart labels.

The project aims to bring some of this technology to the user without having to make a large investment. No need to change the fridge, it has been built thanks to a \emph{Arduino component}, which allows scanner the products.

There are applications that allow control the spending at home, but unlike this project, they are focused to include the total cost of a purchase. This project allows you to scan product by product, and thanks to a collaborative data base, obtain the total cost of the purchase; distinguished by type of food... Also eating habits, create a shopping list, etc.

This solution was divided into three main blocks:

 \begin{description}
    \item Electronic Component

        It is part of the more physical project which has created a device that allows reading the products.

        To do this it has two input modes; through barcode or an RFID tag, and a wireless output to communicate to the server.

    \item Centralized Server

        It is the logic of the project, responsible for receiving the data from the electronic component, store it and server it to client applications.

    \item Client Application

            It is the communication part where all the information is consumed by the user through graphical summaries, lists, etc.

            By crossing data generated by the electronic component and all information contained on the server will show to the customer the following information:
            \begin{itemize}
                \item Current Stock of the fridge
                \item Consumption statistics; lists and statistics with the products purchased using filters like; product, dates, prices...
                \item Expenditure statistics
            \end{itemize}

            By using a centralized server will allow the creation of tools for collaboration between users of the application for the creation and management:

            \begin{itemize}
                \item Product identification; name, price, image, etc.
                \item Purchase plans; allow share purchase plans that can be adjusted to achieve different purposes like saving money.
                \item Recipes; this way you can see based on the stock that you have in the fridge which recipes can be prepared.
            \end{itemize}
\end{description}