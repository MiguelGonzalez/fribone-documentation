\chapter*{Resumen}

El objetivo del proyecto es resolver un problema muy común en el hogar, el control de los gastos a la hora de llenar el frigorífico debido a una mala gestión de las compras. Además, se obtienen otros beneficios como; control de una dieta sana y equilibrada, evitar el desperdicio de alimentos que no se consumen, listas automáticas de los productos a comprar, estadísticas en tiempo real, etc.

Esta solución se divide en tres bloques principales:

 \begin{description}
    \item Componente electrónico

    Es la parte del proyecto más física donde se deberá ensamblar un aparato que permita la lectura de productos.

    Para ello se contará con dos tipos de entrada de productos; a través del código de barras ó un lector RFID, y de una salida Wireless para comunicar al servidor la entrada de los productos.

    \item Servidor centralizado

    Es la parte lógica del proyecto, el encargado de recibir la información del componente electrónico, almacenarla y operar con ella para ofrecer a las aplicaciones cliente acceso a toda la información.

    \item Aplicación cliente

    Es la parte de comunicación donde toda la información es consumida por el usuario a través de gráficas, resúmenes, listados, etc.

    Mediante el cruce de datos generado por el componente electrónico y toda la información contenida en el servidor se le mostrará al cliente la siguiente información:
        \begin{itemize}
        \item Stock actual del frigorífico
        \item Estadísticas de consumo; listados y estadísticas con los productos que se han comprado mediante el uso de filtros como; tipo de productos, fechas, precios...
        \item Estadísticas de gastos
        \end{itemize}

    Al emplear un servidor centralizado se va a permitir la creación de herramientas de colaboración entre los usuarios de la aplicación para la administración y creación de:
        \begin{itemize}
        \item Identificación de productos; nombre, precio, imagen, etc.
        \item Planes de compra; permitirá compartir planes de compra que pueden orientarse a distintos fines como el conseguir un ahorro económico.
        \item Recetas; de esta manera se podrá consultar en base al stock que se dispone en el frigorífico que recetas de cocina se pueden preparar.
        \end{itemize}
\end{description}