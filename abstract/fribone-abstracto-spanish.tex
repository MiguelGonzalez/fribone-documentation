\chapter*{Resumen}

Compañías como LG han presentado en los últimos años frigoríficos inteligentes; frigoríficos que se pueden controlar a través del móvil y permiten conocer los productos que hay en su interior a través de etiquetas inteligentes.

El objetivo del proyecto es traer parte de esta tecnología al usuario sin tener que realizar una gran inversión. No será necesario cambiar de frigorífico, para ello se ha creado un componente gracias a \emph{Arduino}, el cual permite escamear los productos que se introduzcan o se consuman del frigorífico.

Existen aplicaciones que permiten llevar el gasto en el hogar, pero a diferencia de este proyecto, están enfocadas para insertar el gasto total de una compra. Este proyecto permite escanear producto por producto, y gracias a una base de datos colaborativa, obtener el gasto total de la compra; distinguir por tipo de alimento... También se podrá consultar los hábitos de alimentación, consultar que productos faltan, generar una lista de la compra, etc.

Esta solución se divide en tres bloques principales:

 \begin{description}
    \item Componente electrónico

    Es la parte del proyecto más física donde se ha creado un aparato que permite la lectura de productos.

    Para ello se cuenta con dos modos de entrada de productos; a través del código de barras ó de una etiqueta RFID, y de una salida Wireless para comunicar al servidor la entrada de los productos.

    \item Servidor centralizado

    Es la parte lógica del proyecto, el encargado de recibir la información del componente electrónico, almacenarla y operar con ella para ofrecer a las aplicaciones cliente acceso a toda la información.

    \item Aplicación cliente

    Es la parte de comunicación donde toda la información es consumida por el usuario a través de gráficas, resúmenes, listados, etc.

    Mediante el cruce de datos generado por el componente electrónico y toda la información contenida en el servidor se le mostrará al cliente la siguiente información:
        \begin{itemize}
        \item Stock actual del frigorífico
        \item Estadísticas de consumo; listados y estadísticas con los productos que se han comprado mediante el uso de filtros como; tipo de productos, fechas, precios...
        \item Estadísticas de gastos
        \end{itemize}

    Al emplear un servidor centralizado se va a permitir la creación de herramientas de colaboración entre los usuarios de la aplicación para la administración y creación de:
        \begin{itemize}
        \item Identificación de productos; nombre, precio, imagen, etc.
        \item Planes de compra; permitirá compartir planes de compra que pueden orientarse a distintos fines como el conseguir un ahorro económico.
        \item Recetas; de esta manera se podrá consultar en base al stock que se dispone en el frigorífico que recetas de cocina se pueden preparar.
        \end{itemize}
\end{description}