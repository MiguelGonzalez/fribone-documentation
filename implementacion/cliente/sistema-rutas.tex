\subsection{Sistema de rutas}

La navegación en la aplicación web se realiza a través de un sistema de rutas utilizando para ello la librería JavaScript \emph{Director}. Esta librería funciona tanto desde lado cliente como desde el lado servidor con \emph{Node.js}, de hecho, forma parte del framework \emph{flatiron} para desarrollo web utilizando como tecnología \emph{Node.js}.

Con \emph{Director} podemos definir un sistema de rutas, y de rutas anidadas, que permiten ejecutar una (o varias) funciones al encontrarnos sobre ella.

Al cargar la página, se crea un objeto del tipo Router que ofrece \emph{Director}, se le pasa en el constructor nuestro sistema de rutas y automáticamente buscará aquellas rutas que concuerden con la dirección web del navegador para llamar a las funciones que cumplan las condiciones.

Además, para realizar la navegación entre las distintas secciones de la página, se utiliza objeto \emph{Router} para indicarle a donde queremos ir (función \emph{setRoute}). Automáticamente se encargará de modificar la URL y de llamar a las funciones que se hayan definido para dicha ruta.

Para este proyecto se ha configurado router para utilizar la HTML5 History API, si no se indicase utilizaría un sistema de Hashes.

A continuación se muestra el sistema de rutas que se ha definido para el portal privado del usuario:

    \begin{lstlisting}
var routes = {
    '/tablon': tablon.draw,
    '/fridge/:id': {
        '/productos': fridge.draw_productos,
        '/compras': {
            '/:id': fridge.draw_compra,
            on: fridge.draw_compras
        },
        on: fridge.draw
    },
    '/supermercados' : {
        '/:id': supermercado.draw,
        on: supermercados.draw
    }
};
    \end{lstlisting}

Al construir el objeto \emph{Router} se le pasan las rutas que hemos definido y la configuración específica que se desea establecer (\emph{HTML5 History API}).

    \begin{lstlisting}
var router = new Router(routes).configure({
    html5history: true
});
    \end{lstlisting}

Existe otro parámetro de configuración muy interesante, el parámetro \emph{recursion}, por defecto está deshabilitado en el lado cliente, si se activa se puede indicar el orden de recursión; \emph{backward} ó \emph{forward}. La recursión se utiliza cuando se crean subrutas, si se habilita, se puede definir si es llamado primero el hijo y luego el padre (ó padres); o viceversa.

Se ha dejado deshabilitado esta funcionalidad debido a que se controlan las llamadas entre rutas, subrutas... Desde los controladores, lo cual permite controlar si hay que renderizar el padre o si ya está renderizado.

Por último, para que \emph{Router} empiece a funcionar, hay que llamar a su método \emph{init}.

    \begin{lstlisting}
router.init();
    \end{lstlisting}