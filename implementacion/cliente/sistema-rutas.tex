\subsection{Sistema de rutas}

La navegación en la aplicación web se realiza a través de un sistema de rutas utilizando para ello la librería JavaScript \emph{Director}. Esta librería funciona tanto desde lado cliente como desde el lado servidor con \emph{Node.js}, de hecho, forma parte del framework \emph{flatiron} para desarrollo web utilizando como tecnología \emph{Node.js}.

Con \emph{Director} podemos definir un sistema de rutas, y de rutas anidadas, que permiten ejecutar una (o varias) funciones al encontrarnos sobre ella.

Una vez se carga la página web lo primero que realizamos es la creación de un objeto del tipo \emph{Router}, el cual nos lo da la librería. El objeto \emph{Router} recibe en su constructor el sistema de rutas, de esta manera, se encarga automáticamente de llamar a las funciones que concuerden con la ruta que coincida con la dirección web del navegador. También puede recibir otros parámetros opcionales de configuración, en nuestro caso le hemos indicado que utilice la \emph{HTML5 History API} en lugar del sistema de Hashes que emplea por defecto.

El objeto \emph{Router} también tiene un método para indicarle a que zona queremos navegar, para ello se llama a \emph{setRoute} pasando como parámetro la \emph{URI}. El objeto \emph{Router} se encargará de modificar la \emph{URI} y de llamar a las funciones que se hayan definido para dicha ruta.

A continuación se muestra el sistema de rutas que se ha definido para el portal privado del usuario:

    \begin{lstlisting}
var routes = {
    '/tablon': tablon.draw,
    '/fridge/:id': {
        '/productos': fridge.draw_productos,
        '/compras': {
            '/:id': fridge.draw_compra,
            on: fridge.draw_compras
        },
        on: fridge.draw
    },
    '/supermercados' : {
        '/:id': supermercado.draw,
        on: supermercados.draw
    }
};
    \end{lstlisting}

Al construir el objeto \emph{Router} se le pasan las rutas que hemos definido y en la configuración se activa el uso del \emph{HTML5 History API}.

    \begin{lstlisting}
var router = new Router(routes).configure({
    html5history: true
});
    \end{lstlisting}

Existe otro parámetro de configuración muy interesante, el parámetro \emph{recursión}, por defecto está deshabilitado en el lado cliente, si se activa se puede indicar el orden de recursión; \emph{backward} ó \emph{forward}. La recursión se utiliza cuando se crean subrutas, si se habilita, se puede definir si es llamado primero el hijo y luego el padre (o padres); o viceversa.

Se ha dejado deshabilitado esta funcionalidad debido a que se controlan las llamadas entre rutas, subrutas... Desde los controladores, lo cual permite controlar si hay que renderizar el padre o si ya está renderizado.

Por último, para que \emph{Router} empiece a funcionar, hay que llamar a su método \emph{init}.

    \begin{lstlisting}
router.init();
    \end{lstlisting}