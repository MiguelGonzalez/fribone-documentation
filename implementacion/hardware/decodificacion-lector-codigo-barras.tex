\subsection{Decodificando el lector de código de barras}

El siguiente problema a resolver es la creación de un script en \emph{python} que permita la lectura de códigos de barras. Para ello se ha tenido que decodificar los caracteres que envía al sistema operativo, y una vez hecho, comprender como funciona.

El funcionamiento es muy sencillo. Cuando el lector de código de barras encuentra un código válido lo envía al sistema por letras y por último, y es lo que nos ayuda a comprender que ha terminado de enviarnos el código de barras, nos envía dos caracteres seguidos (\emph{Control + j}) que simbolizan el salto de línea (\emph{Line Feed}).

El siguiente script permite leer los caracteres escaneados e imprimirlos por consola o enviarlos al microcontrolador a través de la librería \emph{Bridge}.

\begin{lstlisting}
#!/usr/bin/python
# -*- coding: utf-8 -*-

import struct
import time
import sys
sys.path.insert(0, '/usr/lib/python2.7/bridge/')

from bridgeclient import BridgeClient as bridgeclient

# Variables
infile_path = "/dev/input/event1"

#long int, long int, unsigned short, unsigned short, unsigned int
FORMAT = 'llHHI'
EVENT_SIZE = struct.calcsize(FORMAT)

teclasPulsadas = dict()
escribiendoLetras = False
codigoBarras = ''
bridgeCliente = bridgeclient()
in_file = None

# Funciones

def esTeclaNumerica(value):
    return value >= 458782 and value <= 458791

def getNumero(value):
    return (value - 458781)%10

def esTeclaLetra(value):
    return (value >= 458756 and value <= 458781)

def getLetra(value):
    posicionValue = value - 458756
    asciiValue = posicionValue + 65
    return str(unichr(asciiValue))

# 458792 -> Intro
# 458976 + 458765 (Control + j => Line Feed)
def esTeclaControl(value):
    return value == 458792 or \
            value == 458976 or \
            value == 458765
def procesarEvento(event):
    global teclasPulsadas, escribiendoLetras, codigoBarras, bridgeCliente
    (tv_sec, tv_usec, type, code, value) = struct.unpack(FORMAT, event)

    # Importa este orden de la condicion, si se altera puede imprimir
    # caracteres escritos muy deprisa en otro orden
    if value not in teclasPulsadas:
        teclasPulsadas[value] = 'S'
    else:
        if esTeclaControl(value) and escribiendoLetras:
            escribiendoLetras = False
            print codigoBarras
            #bridgeCliente.put('codebar', codigoBarras)

            codigoBarras = ''
        elif not esTeclaControl(value) and \
                (esTeclaNumerica(value) or \
                esTeclaLetra(value)):
            escribiendoLetras = True
            if esTeclaNumerica(value):
                codigoBarras += str(getNumero(value))
            elif esTeclaLetra(value):
                codigoBarras += getTeclaLetra(value)
        del teclasPulsadas[value]

def conectarUsb():
    global in_file
    in_file = None
    conectado = False
    while not conectado:
        try:
            # Intentamos abrir el fichero
            in_file = open(infile_path, "rb")
            conectado = True
        except IOError:
            time.sleep(1) # Pausamos 1seg

# Inicio programa
conectarUsb()
event = in_file.read(EVENT_SIZE)
while event:
    try:
        procesarEvento(event)
        event = in_file.read(EVENT_SIZE)
    except IOError:
        conectarUsb()
        event = in_file.read(EVENT_SIZE)
\end{lstlisting}