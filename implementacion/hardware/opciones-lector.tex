\subsection{Opciones del lector}

El lector cuenta con una pequeña \emph{pantalla TFT} y unos pocos botones para su manejo. La primera opción que se da al lector es su vinculación con la cuenta del usuario y para ello se muestra el siguiente aviso en la \emph{pantalla TFT}:

Una vez se ha vinculado, escaneando para ello los dos códigos de barras generados por la aplicación web, se muestra la siguiente pantalla con las opciones disponibles:

[FOTOGRAFIA PANTALLA INICIO]

Para acceder a cada opción de menú hay que pulsar cada botón correspondiente; superior para listar los frigoríficos, inferior para desvincular el producto, y laterales para añadir o o quitar productos del frigorífico.

Al acceder a cada opción de menú se muestra una pantalla diferente, con una opción para \emph{Volver} y opcionalmente alguna otra u otras opciones:

\begin{itemize}

    \item Selección del frigorífico activo

        Esta opción de menú nos permitirá elegir el frigorífico sobre el que se realizarán las acciones (entrada / salida) de productos.

        [FOTOGRAFIA OPCIÓN DE MENÚ]

        A través del botón inferior podemos ir moviéndonos entre los frigoríficos que dispone el usuario, para ello se han traído a través de una petición \emph{RESTful}. Con el botón derecho podemos Aceptar el frigorífico actual y con el izquierdo regresar a la pantalla anterior.

        En el caso de no contar con frigoríficos se muestra el siguiente mensaje en pantalla:

        [FOTOGRAFIA OPCIÓN DE MENÚ SIN FRIGORÍFICOS]

    \item Modo entrada de productos

        Para introducir productos al frigorífico seleccionamos la opción de menú para \emph{Añadir productos al carrito}, donde se activará la lectura de productos por \emph{código de barras} ó a través del \emph{lector RFID}. En la pantalla se muestra la siguiente información:

        [FOTOGRAFIA OPCIÓN DE MENÚ]

        Al escanear un producto, mientras se realiza la petición \emph{RESTful} para añadirlo al frigorífico, se muestra el siguiente mensaje:

        [FOTOGRAFIA AÑADIENDO PRODUCTO]

        Una vez añadido, se muestra la siguente información:

        [FOTOGRAFIA AÑADIDO OK]

        Al pulsar el botón izquierdo se regresa a la pantalla anterior para acceder a las opciones de menú.

    \item Modo salida de productos

        Para sacar productos del frigorífico nos colocaremos en esta opción del menú el cual activará la lectura de productos por \emph{código de barras} ó a través del \emph{lector RFID}.

        [FOTOGRAFIA OPCIÓN DE MENÚ]

    \item Desvinculación del aparato

        Esta opción de menú borrará de la memoria \emph{MicroSD} la clave pública de acceso, con lo cual se quedará desvinculado el lector de nuestra cuenta de usuario en la aplicación web. Además se lanza una petición \emph{RESTful} para indicar al servidor que se quiere desvincular el lector.

        Antes de iniciar la desvinculación se muestra una pantalla de confirmación al usuario, debiendo pulsar el botón inferior para \emph{Aceptar} la opción:

        [FOTOGRAFIA DESVINCULAR CONFIRMACION]

        Una vez se confirma, se lanza la petición al servidor y se procede a eliminar la clave pública, mientras se muestra el siguiente mensaje en pantalla:

        [FOTOGRAFIA DESVINCULANDO]

        Una vez se ha desvinculado correctamente se vuelve a mostrar la pantalla de vinculación:

        [FOTOGRAFIA VINCULACION]

\end{itemize}