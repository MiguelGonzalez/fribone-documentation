\subsection{Opciones del lector}

Una vez se ha vinculado el lector, utilizando la clave pública de acceso se lanza una petición para obtener la información del cliente. Con esta información, nombre y frigorificos que dispone, se dibuja la pantalla de inicio con el nombre del cliente, el frigorífico activo seleccionado y las opciones de menú.

    [FOTOGRAFIA PANTALLA INICIO]

El dispositivo cuenta con cuatro botones para manejar el menú colocados en disposición de cruceta y dos botones para \emph{Aceptar} ó \emph{Cancelar} una opción.

A través de los botones podremos navegar por las distintas opciones del lector:

\begin{itemize}

    \item Selección del frigorífico activo

        Esta opción de menú nos permitirá elegir el frigorífico sobre el que se realizarán las acciones (entrada / salida) de productos.

        [FOTOGRAFIA OPCIÓN DE MENÚ]

    \item Modo entrada de productos

        Para introducir productos al frigorífico nos colocaremos en esta opción de menú el cual activará la lectura de productos por \emph{código de barras} ó a través del \emph{lector RFID}.

        [FOTOGRAFIA OPCIÓN DE MENÚ]

    \item Entrada de un producto de forma manual

        Debido a las limitaciones del lector de código de barras hay algunos productos que presentan dificultades para ser escaneados, en concreto, aquellos cuyos códigos de barras sean muy pequeños o estén localizados en zonas curvas.

        Por ello, se permite la introducción manual del código de barras utilizando para ello la cruceta:

        [FOTOGRAFIA OPCIÓN DE MENÚ]

    \item Modo salida de productos

        Para sacar productos del frigorífico nos colocaremos en esta opción del menú el cual activará la lectura de productos por \emph{código de barras} ó a través del \emph{lector RFID}.

        [FOTOGRAFIA OPCIÓN DE MENÚ]

    \item Salida de un producto de forma manual

        Al igual que a la entrada de productos se pueden presentar problemas para escanear un producto, sucede lo mismo a la salida del producto, por lo que se da la misma opción para introducir el código de barras de manera manual.

        [FOTOGRAFIA OPCIÓN DE MENÚ]

    \item Desvinculación del aparato

        Esta opción de menú borrará de la memoria \emph{MicroSD} la clave pública de acceso, con lo cual se quedará desvinculado el lector de nuestra cuenta de usuario en la aplicación web.

        Para volver a utilizar el lector será necesario volver a enlazarlo.

        [FOTOGRAFIA OPCIÓN DE MENÚ]

\end{itemize}