\subsection{Problemas detectados}

La versión de Linux que trae la placa Arduino no viene con las librerías para utilizar el puerto USB en modo HID (\emph{Human Interface Device}) por lo que no reconoce el lector de códigos de barras USB. Para configurarlo se han tenido que bajar las librerías e instalarlas a través de SSH.

\begin{enumerate}

    \item Descargar y guardar en la memoria micro-sd (\emph{/mnt/sda1/}) las dos siguientes librerías:

        \begin{itemize}
            \item kmod-hid-generic\_3.8.3-1\_ar71xx.ipk
            \item kmod-hid\_3.8.3-1\_ar71xx.ipk
        \end{itemize}

    \item Actualizar el sistema de paquetes e instalar las dos siguientes librerías para permitir la instalación de las librerías anteriores:

        \begin{lstlisting}
opkg update
opkg install kmod-input-core
opkg install kmod-input-evdev
        \end{lstlisting}

    \item Instalar las librerías de la micro-sd:

        \begin{lstlisting}
rm /tmp/opkg-lists/*
opkg install /mnt/sda1/kmod-hid\_3.8.3-1\_ar71xx.ipk
opkg install /mnt/sda1/kmod-hid-generic\_3.8.3-1\_ar71xx.ipk
        \end{lstlisting}

    \item Instalar el driver HID:

        \begin{lstlisting}
opkg update
opkg install kmod-usb-hid
        \end{lstlisting}

    \item Cargar el driver HID en el Kernel:

        \begin{lstlisting}
insmod hid-generic
echo "hid-generic" >>/etc/modules.d/62-hid-generic
        \end{lstlisting}

    \item Reiniciar Linux:

        \begin{lstlisting}
reboot
        \end{lstlisting}

    \item Enchufar el lector de tarjetas USB y comprobar que funciona:

        \begin{lstlisting}
cat /dev/input/event1 | hexdump
        \end{lstlisting}

\end{enumerate}

Con esta configuración Linux es capaz de cargar el dispositivo USB, pero, si se desenchufa y se vuelve a enchufar a veces no carga y da problemas. Por suerte, el \emph{23 de Abril de 2014} han sacado una nueva versión de \emph{OpenWrt-Yun} que corrige estos errores, e introduce las siguientes mejoras:

    \begin{itemize}
        \item OpenWrt
            \begin{itemize}
                \item wget now supports ssl
                \item Fixed fuses values in run-avrdude
                \item nano is now built-in (no need to become ''vi'' experts any more)
                \item Updated ruby to 1.9.3-p429
                \item Fixed USB port stability (when using it both with a pen drive or as a serial device)
                \item Patched dhcp script to fix ethernet routing issue
                \item Added lots of webcam modules
                \item Fixed USB keyboard support
                \item Heartbleed http://heartbleed.com/
                \item Linux side ready visual notification: when linux boot completes, the usb led lights up (it's bright white)
                \item Disk space expander tutorial: using an micro SD card to have gigabytes of disk space
                \item Added ''upgrade-all'' script for easier massive upgrade of packages (available only after having followed the ExpandingYunDiskSpace tutorial)
                \item Node.js is now available as an optional package: other native nodejs packages include bleno, noble, serialport, socket.io.
            \end{itemize}

        \item Web panel
            \begin{itemize}
                \item Previously, jsonp calls were triggered by the ''jsonp'' query string parameter only. Now, you can use ''callback'' as well.
                \item Added Mailbox support
                \item Various fixes
            \end{itemize}

        \item Bridge
            \begin{itemize}
                \item File.size() now implemented
                \item PHP bridge client (thx Luca Saltoggio)
                \item Bridge is now run with ''-u'' python flag, preventing some random lockups in the Bridge
                \item Resolved conflict with python ''json'' module
            \end{itemize}
    \end{itemize}

Una vez actualizado la versión de \emph{OpenWrt-Yun} siguiendo las instrucciones de la página de Arduino (\emph{http://arduino.cc/en/Tutorial/YunSysupgrade}) solamente tenemos que instalar la librería USB HID por SSH (ya que sigue sin venir por defecto):

    \begin{lstlisting}
opkg update
opkg install kmod-usb-hid
    \end{lstlisting}

Y nos funcionará el lector de códigos de barras USB sin ningún problema, aunque se desenchufe y se vuelva a enchufar.