\subsection{Problemáticas para la creación de tests unitarios}

CodeIgniter viene con su propia herramienta para generar tests unitarios, la cual no nos permite su integración con \emph{Travis CI}, por ello hemos realizados una serie de modificaciones para permitir el uso de PHPUnit.

Se han tenido que tocar los siguientes módulos y realizar los siguientes cambios:

\begin{itemize}

    \item Modificación del fichero \emph{DB\_driver.php}

        Si se produce un error se llama al método display\_error, el cual lo imprime por pantalla. Para nuestros tests se ha modificado para lanzar una excepción en caso de que estemos en el entorno de testing.

                \begin{lstlisting}
if (defined('ENVIRONMENT') && ENVIRONMENT == 'testing') {
    $message = $error . ' ' . $swap;
    throw new Exception($message);
} else {
    ...
    ...
}
                \end{lstlisting}

    \item Modificación del fichero Utf8.php

        Se ha modificado la línea donde se declara la variable global \emph{\$CFG} por el siguiente código para evitar un error al lanzarse los tests:

                \begin{lstlisting}
//global $CFG; // Fix PhpUnit Error
$CFG =& load_class('Config', 'core');
                \end{lstlisting}

    \item Modificado fichero \emph{mysql\_driver.php} para evitar un error al utilizar la función \emph{mysql\_escape\_string}:

                \begin{lstlisting}
function escape_str($str, $like = FALSE) {
        if (is_array($str)) {
            foreach ($str as $key => $val) {
                $str[$key] = $this->escape_str($val, $like); }

            return $str;
        }

        $str = is_resource($this->conn_id) ? mysql_real_escape_string($str, $this->conn_id) : addslashes($str);

        // escape LIKE condition wildcards
        if ($like === TRUE) {
            return str\_replace(
                array($this->_like_escape_chr, '%', '_'),
                array($this->_like_escape_chr.$this->_like_escape_chr,
                    $this->_like_escape_chr.'%', $this->_like_escape_chr.'_'),
                $str);
        }

        return $str;
    }
                \end{lstlisting}

\end{itemize}

Por último, para ser capaces de lanzar nuestros tests hemos tenido que añadir las siguientes configuraciones a nuestra aplicación:

\begin{itemize}
    \item Creación del fichero de configuración \emph{phpunit.xml}

        Este fichero XML nos permite indicarle a PHPUnit los parámetros para ejecutar los tests. Consta de la siguiente configuración:

                \begin{lstlisting}
<?xml version="1.0" encoding="UTF-8" ?>
<phpunit bootstrap="application/application/tests/bootstrap.php"
    colors="true"
    convertErrorsToExceptions="true"
    convertNoticesToExceptions="true"
    convertWarningsToExceptions="true"
    processIsolation="false"
    stopOnFailure="false"
    syntaxCheck="false"
    verbose="true">
    <testsuites>
        <testsuite name="TestSuite">
            <directory>application/tests</directory>
        </testsuite>
    </testsuites>
    <php>
        <const name="PHPUNIT_TEST" value="1" />
        <const name="PHPUNIT_CHARSET" value="UTF-8" />
        <const name="REMOTE_ADDR" value="217.0.0.1" />
    </php>
    <filter>
        <blacklist>
            <directory suffix=".php">system</directory>
        </blacklist>
    </filter>
</phpunit>
                \end{lstlisting}

        Ejecutará nuestra aplicación a partir del fichero \emph{bootstrap.php} y lanzará los tests contenidos en la carpeta \emph{application/tests}.

    \item Creación del fichero \emph{bootstrap.php}

        Este fichero es una copia del \emph{index.php}, se ha modificado el entorno a testing, la dirección remota del servidor, el reporte de errores del sistema y el path de las carpetas:
                \begin{lstlisting}
define('ENVIRONMENT', 'testing');

$_SERVER["REMOTE_ADDR"]     = array_key_exists( 'REMOTE_ADDR', $_SERVER) ? $_SERVER['REMOTE_ADDR'] : '127.0.0.1';

error_reporting(E_ALL ^ E_NOTICE);
$system_path = '../../system';
$application_folder = '../../application';
                \end{lstlisting}

    \item En nuestra definición de controlador global se ha añadido una condición para comprobar si estamos en modo testing. En modo testing no se carga ninguna librería y se añade en su lugar la carpeta de \emph{tests/mockups} como paquete, de esta manera se cargarán en nuestros tests de ahí aquellas librerías mockups que se añadan.

                \begin{lstlisting}
if (defined('ENVIRONMENT') && ENVIRONMENT == 'testing') {
    $this->load->add_package_path(APPPATH.'tests/mockups');
} else {
    $this->load->library('login_auth');
}
                \end{lstlisting}
\end{itemize}