\chapter{LÍNEAS FUTURAS}

La principal motivación de este proyecto es permitir a los hogares contar con un dispositivo económico para cuantificar el consumo alimenticio, es decir, guardar toda la información sobre las compras realizadas, cuando se ha consumido un alimento, qué es necesario comprar, etc.

Hoy en día no hay soluciones baratas a este problema ya que las grandes compañías están aplicando esta tecnología a la creación de frigoríficos futuristas. Con este proyecto se pretende lograr aplicar esta tecnología futurista a través de un componente sencillo y barato, soportado por una aplicación que se encargue de ofrecer toda la información detallada.

Con el objetivo de poder comercializar realmente este proyecto se proponen las siguientes líneas futuras.

\section{COMPONENTE HARDWARE}

El componente hardware que se ha realizado se puede considerar un prototipo, para una comercialización del mismo se deberían realizar las siguientes mejoras:

\begin{itemize}
    \item Diseñar e integrar un lector de código de barras en miniatura dentro de la caja.
    \item Integrar una pantalla más grande y con respuesta táctil para ahorrarnos los botones.
    \item Diseñar una caja en plástico para contener todos los componentes en su interior
\end{itemize}

\section{COMPONENTE SERVIDOR Y SOFTWARE}

Ahora mismo solo se permite el registro a través del propio servidor lo cual implica una molestia al usuario al tener que rellenar sus datos para registrarse en un servicio una vez más. Una propuesta a mejorar es el registro través de otros servicios como \emph{Google} o \emph{Facebook}.

Por otro lado, se dispone de una base de datos que los usuarios hacen crecer en base a sus compras. Con toda esta información, se pueden realizar soluciones de \emph{Data Mining} que le ofrezcan a los usuarios información útil.

Por último, para obtener beneficio económico de la plataforma sería interesante negociar con cadenas alimenticias para ofrecer compras a través del servidor, con envio a domicilio.