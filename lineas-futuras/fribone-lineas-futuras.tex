\chapter{LÍNEAS FUTURAS}

La principal motivación de este proyecto es acercar la tecnología a una de las actividades cotidianas que menos ha cambiado a lo largo de la historia; realizar la compra.

Para ello se ha diseñado un dispositivo que permita cuantificar y ofrecer información al usuario acerca de sus compra. Por ejemplo, permitir consultar los productos que dispone en el frigorífico, las compras que ha realizado a lo largo del tiempo, análisis de gastos, etc.

Las grandes compañías a nivel mundial aún no han dado el salto a utilizar este tipo de tecnologías en sus productos convencionales. Sí disponen de algun frigorífico de gama alta con tecnología futurista, los cuales, no son accesibles para el consumidor medio.

Por ello, con este proyecto se pretende acercar al consumidor a parte de esta tecnología futurista. A través de un componente sencillo, barato y elegante. Así, con el objetivo de poder comercializar realmente este proyecto se proponen las siguientes líneas futuras.

\section{COMPONENTE HARDWARE}

El componente hardware que se ha diseñado es un prototipo funcional del producto final. Pero, para una comercialización del producto es recomendable realizar las siguientes innovaciones:

\begin{itemize}
    \item Miniaturización

        El componente que más ocupa del prototipo es el lector de código de barras. Es necesario encontrar un proveedor que proporcione un lector pequeño y discreto, de esta manera se puede integrar en la parte inferior del producto sin ocupar un gran espacio.

    \item Diseño

        Fribone es un producto pensado para el hogar, en concreto, la cocina. Por lo que tiene que ser un producto elegante e integrador, nadie quiere un producto que rompa la armonía de la cocina.

    \item Ergonomía

        La mayoría de las personas experimentan algún grado de limitación física en algún momento de su vida. Por lo tanto, el diseño final del producto, además de ser elegante, debe estar pensado para que pueda ser manejado por personas con ciertas discapacidades físicas.

        Una manera de afrontar esta tarea es pensar en la eliminación de los botones del aparato, que pueden ser dificiles de pulsar, y añadir una pantalla más grande y con respuesta táctil. Así, se puede realizar una interfaz con unos botones en pantalla grandes y visibles.

\end{itemize}

\section{COMPONENTE SERVIDOR Y SOFTWARE}

La arquitectura y diseño del software ha seguido una metodología ágil. Así, se ha conseguido levantar una infraestructura usable en un tiempo breve. Gracias a los tests unitarios y a la plataforma de integración continua, \emph{TravisCI}, se pueden seguir desarrollando pequeños cambios incrementales para dotar de más potencia a la aplicación:

\begin{itemize}

    \item Universalidad

        Integrar soluciones de acceso a nuestra plataforma a partir de otros servicios (\emph{Facebook}, \emph{Google}, etc.), y ofrecer una \emph{API RESTful} basada en \emph{OAuth 2} para la autorización del acceso a la información.

    \item Data Mining

        Explorar y buscar nuevas técnicas de análisis de la información que ingresan los usuarios para poder ofrecerles sugerencias en base a sus gustos, consejos a la hora de comprar, etc.

    \item Servicios a terceros

        Animar a las compañías del sector comercial a ofrecer sus productos a través de la plataforma. De esta manera, se le pueden ofrecer nuevas experiencias al usuario como la planificación de compras automáticas en base a los productos que dispone en su frigorífico.
    \end{itemize}