\section{Requisitos funcionales}

Los requisitos funcionales se expondrán en forma de casos de uso.

\subsection{Caso de uso Escaneo de un producto}

\textbf{Actor principal.} Usuario.

\textbf{Actores de apoyo.} Ninguno.

\textbf{Precondiciones.} El producto debe tener un código de barras legible. El componente hardware debe estar configurado y operativo.

\textbf{Postcondiciones.} Ninguna.

\textbf{Escenario principal de éxito.} El sistema detecta correctamente el código de barras y lo transmite al servidor vía WiFi.

\textbf{Escenario alternativo.} El sistema no reconoce el código de barras y se debe introducir a mano en el sistema.

\textbf{Cómo probarlo.} El usuario deberá escanear un producto y comprobar que aparece en la aplicación cliente.

\subsection{Caso de uso Obtener información del frigorífico}

\textbf{Actor principal.} Usuario.

\textbf{Actores de apoyo.} Ninguno.

\textbf{Precondiciones.} Tener acceso a la aplicación cliente y estar logueados en el sistema.

\textbf{Postcondiciones.} Ninguna.

\textbf{Escenario principal de éxito.} El sistema recupera la información de productos del servidor y nos muestra la información.

\textbf{Cómo probarlo.} El usuario se identifica en la aplicación y debe ser capaz de ver la información del frigorífico.
